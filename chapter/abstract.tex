\begin{cnabstract}
量子色动力学是用来描述夸克和胶子间强相互作用的规范场理论。格点量子色动力学预言在高温或高重子化学势的条件下会发生从强子物质到夸克胶子等离子体(QGP)的相变。坐落在美国布鲁克海文国家实验室(BNL)的相对论重离子对撞机(RHIC)是专门用于研究夸克胶子等离子体性质以及量子色动力学相图的实验装置。双轻子不参与强相互作用,并且可以在重离子对撞整个演化过程中产生,因此,双轻子的测量在研究这种高温高密物质中起着至关重要的作用。

根据不同的产生机制,双轻子的不变质量谱一般被划分成三个质量区间。高质量区间 (HMR, $M_{ll}>M_{J/\psi}$), 双轻子主要由初始的硬过程产生,例如 Drell-Yan,夸克偶素的衰变。中间质量区间(IMR, $M_{\phi}<M_{ll}<M_{J/\psi}$), 双轻子主要由夸克胶子等离子体的热辐射以及开粲的半轻子衰变产生,其中热辐射的双轻子产额可用于测量夸克胶子等离子体的温度。低质量区间(LMR, $M_{ll}<M_{\phi}$),双轻子主要由在强子介质中矢量介子($\rho$, $\omega$, $\phi$, 等)的衰变产生, 他们可用于研究介质中的手征对称性恢复。此外,ALICE合作组最近观察到在质心能量为2.76 TeV的铅核-铅核偏心对撞中,超低横动量($p_{T}<$ 0.3 GeV/$c$)的前向快度$J/\psi$产额有非常大的增强。这部分增强有可能来自相干光产生过程。如果在偏心重离子对撞中,也可以通过相干光产生生成$\rho$介子, 这部分$\rho$介子可用作一个直接测量夸克胶子等离子体性质的探针。

本论文利用位于相对论重离子对撞机上的螺旋径迹探测器(STAR),首次研究了双轻子在铀核-铀核对撞中的产生。 用于该分析研究的数据采集于2012年。利用时间投影室测量的电离能损以及飞行时间探测器测量的粒子速度进行正负电子的鉴别。在铀核-铀核的最小无偏对撞中(中心度:0-80\%),鉴别出来的电子整体纯度可以达到95\%。通过对比在最小无偏对撞中测量的STAR接收度内($p_{T}^{e}$ > 0.2 GeV/$c$, $|\eta^{e}|<$ 1, and $|y_{ee}|<$ 1)的双轻子不变质量谱和不包含$\rho$介子贡献的强子衰变模拟(cocktail),我们发现在类$\rho$质量区间0.3-0.76 GeV/$c^{2}$内,测量的双轻子产额比模拟的产额高 2.1 $\pm$ 0.1(stat.) $\pm$ 0.2(sys.) $\pm$ 0.3(cocktail) 倍。我们还系统的测量了不同横动量以及中心度区间的双轻子不变质量谱,发现此增强因子并没有很强的中心度以及横动量依赖性。为了定量的研究这些双轻子增强,我们还测量了修正STAR接收度的双轻子增强谱(data - cocktail)。上面提到的所有双轻子增强谱都可以用一个包含$\rho$展宽的谱函数以及夸克胶子等离子体热辐射贡献的理论模型描述。$\rho$介子谱在高温高密介质中的展宽被认为和手征对称性恢复有关。进一步的分析研究表明,带电粒子密度($dN_{ch}/dy$)归一的修正了STAR接收度的积分增强产额(积分区间:0.4 $<M_{ee}<$ 0.75 GeV/$c^{2}$)有很强的中心度以及对撞能量的依赖性。中心对撞中的归一积分增强产额比偏心对撞以及低能量对撞的产额要高。最近一个理论模型指出,在质心能量为6到200 GeV区间内,$dN_{ch}/dy$归一的积分增强产额正比于重离子对撞中产生介质的寿命。这预示着在铀核-铀核中心对撞中产生的介质的寿命比在偏心对撞中或者低质心能量重离子对撞中产生的介质寿命长。

本论文还首次测量了铀核-铀核对撞中STAR接收度内超低横动量($p_{T}<$ 0.15 GeV/$c$)的双轻子不变质量谱。相对于强子衰变的模拟产额,偏心对撞中的双轻子产额在整个质量区间都有很大的增强。在质量区间0.4 - 0.76 GeV/$c^{2}$和2.8 - 3.2 GeV/$c^{2}$中,增强因子分别为16.4 $\pm$ 1.1(stat.) $\pm$ 2.6(sys.) $\pm$ 4.2(cocktail),20.4 $\pm$ 4.2(stat.) $\pm$ 3.0(sys.) $\pm$ 3.2(cocktail)。这些增强可能来自于相干光产生过程。我们还测量了铀核-铀核对撞中STAR接收度内不同质量区间的双轻子横动量谱(0.4 $<M_{ee}<$ 0.76 GeV/$c^{2}$, 1.2 $<M_{ee}<$ 2.67 GeV/$c^{2}$, and 2.8 $<M_{ee}<$ 3.2 GeV/$c^{2}$),发现这些横动量谱的形状在偏心对撞中在0.1 GeV/$c$附近发生急剧变化。

此外,本论文还报告了两种气体探测器-迷你漂移厚气体电子倍增室(mini-drift THGEM)和多气隙阻性板室(MRPC)的研制以及测试结果。THGEM用作穿越辐射探测器(TRD)的读出探测器,用来鉴别电子离子对撞机上的前向散射电子和提供额外的电离能损($dE/dx$)测量,后者对小角度散射的带电粒子径迹重建非常重要。这是首次提出用THGEM作为TRD的读出探测器。宇宙线测试结果表明,在工作电压下,THGEM的探测效率高于94\%, 位置分辨能达到220 $\mu$m。 由于THGEM具有非常好的位置分辨以及相对厚的电离区,THGEM展现出非常卓越的径迹重建能力。 最后,测试结果表明THGEM增益均匀性以及稳定性也非常好。为了提高北京谱仪的粒子鉴别能力,MRPC被用来升级北京谱仪端盖飞行时间探测器(eTOF)。我们在正负电子对撞机E3束流线上用动量为600 MeV/$c$的质子束测试了单端读出和双端读出MRPC。 在工作高压下,两种MRPC的探测效率都高于98\%。单端读出MRPC的时间分辨为47 ps,但有带电粒子入射位置的依赖性。双端读出MRPC的时间分辨为40 ps,且没有带电粒子入射位置的依赖性。根据这次束流测试结果,双端读出MRPC被用于北京谱仪的eTOF升级。北京谱仪的eTOF升级已于2015年11月完成,对电子的时间分辨可达到60 ps,远好于其设计指标(80 ps)。

%\keywords{中国科学技术大学\enskip 学位论文\enskip \LaTeX{}~通用模板\enskip 学士\enskip 硕士\enskip 博士}
\end{cnabstract}

\begin{enabstract}
Quantum Chromodynamics (QCD) is a basic gauge field theory to describe strong interactions, a fundamental force describing the interactions between quarks and gluons. Lattice QCD calculations predict a smooth cross-over transition from hadronic phase to the Quark Gluon Plasma (QGP) phase at high temperature ($T$) and small baryon chemical potential ($\mu_{B}$), and a first order phase transition at large $\mu_{B}$ region. The Relativistic Heavy Ion Collider, located at Brookhaven National Laboratory, is a dedicated machine to study the properties of the QGP and the QCD phase diagram in laboratory. Dileptons are produced in the whole evolution of the system and escape with minimum interaction with the strongly interacting medium. Thus, dilepton measurements play an essential role in the study of hot and dense nuclear matter.

According to different physics interests, the dilepton invariant mass spectrum is typically divided into three regions. The High Mass Region (HMR, $M_{ll}>M_{J/\psi}$ ), in which the dileptons are produced by the initial hard perturbative QCD process (Drell-Yan, quarkonia etc.). The Intermediate Mass Region (IMR, $M_{\phi}<M_{ll}<M_{J/\psi}$), in which the dileptons are expected to be directly related to the thermal radiation of the QGP which can be used to determine the initial temperature of the QGP medium. The Low Mass Region (LMR, $M_{ll}<M_{\phi}$), in which the dilepton production is dominated by the in-medium decay of vector mesons ($\rho$, $\omega$, $\phi$, etc.) which are considered as a link to chiral symmetry restoration. Recently, a significant excess of J/$\psi$ yield at very low $p_{T}$ ($p_{T}<$ 0.3 GeV/$c$) has been observed by the ALICE collaboration in peripheral hadronic Pb + Pb collisions at $\sqrt{s_{NN}}$ = 2.76 TeV at forward-rapidity, which may be related to coherent photoproduction of J/$\psi$. If $\rho$ meson can be produced via coherent photoproduction process in peripheral heavy-ion collisions, it might sit in the QGP before its decay. This provides a direct probe of the QGP.

In this thesis, we report the measurements of dielectron production in U + U collisions at $\sqrt{s_{NN}}$ = 193 GeV at Solenoidal Tracker at RHIC (STAR). The data set used in this analysis was taken in year 2012 (Run12). The electron (positron) can be identified by combing ionization energy loss $dE/dx$ measured by the Time Projection Chamber (TPC) and particle velocity measured by the Time of Flight (TOF). A $\sim$95\% overall electron purity can be achieved in U + U minimum-bias collision (0-80\%) at $\sqrt{s_{NN}}$ = 193 GeV. The measured dielectron invariant mass spectrum within STAR acceptance ($p_{T}^{e}$ > 0.2 GeV/$c$, $|\eta^{e}|<$ 1, and $|y_{ee}|<$ 1) in minimum-bias collisions shows an enhancement with respect to hadronic cocktail simulation in 0.3 $<M_{ee}<$ 0.76 GeV/$c^{2}$ ($\rho$-like). The enhancement factor (data/cocktail), integrated over the $\rho$-like mass region and the full $p_{T}$ acceptance, is 2.1 $\pm$ 0.1(stat.) $\pm$ 0.2(sys.) $\pm$ 0.3(cocktail). Meanwhile, systematic measurements are performed in differential $p_{T}$ and centrality bins. The enhancements in LMR are consistently observed and the enhancement factor has a mild $p_{T}$ or centrality dependence. To quantitatively study the dielectron excess, the acceptance-corrected dielectron excess mass spectrum (data - cocktail) is also performed. These excess spectra can be consistently described by a theoretical model calculation incorporating a broadened $\rho$ spectral function and QGP thermal radiation. The integrated acceptance-corrected excess yield in 0.4 $<M_{ee}<$ 0.75 GeV/$c^{2}$ normalized by charged particle density ($dN_{ch}/dy$), has a strong centrality and collision-energy dependence. Recently, it is found in a model calculation that the $dN_{ch}/dy$ normalized dilepton excess yield in the low mass region is proportional to the lifetime of the hot, dense medium created in heavy-ion collisions at $\sqrt{s_{NN}}$ = 6 - 200 GeV. The $dN_{ch}/dy$ normalized integrated yield of the most central U + U collisions at $\sqrt{s_{NN}}$ = 193 GeV is higher than those in peripheral or lower-energy collisions, which indicates that the hot and dense medium created in central U + U collisions at $\sqrt{s_{NN}}$ = 193 GeV has a longer lifetime.

We also report the dielectron invariant mass spectra within STAR acceptance at very low $p_{T}$ ($p_{T}<$ 0.15 GeV/$c$) in U + U collisions. The dielectron invariant mass spectrum shows a significant enhancement compared to hadronic cocktail for the entire mass region in the most peripheral (60-80\%) collisions. The enhancement factors are 16.4 $\pm$ 1.1(stat.) $\pm$ 2.6(sys.) $\pm$ 4.2(cocktail) and 20.4 $\pm$ 4.2(stat.) $\pm$ 3.0(sys.) $\pm$ 3.2(cocktail) in 0.4 $<M_{ee}<$ 0.76 GeV/$c^{2}$ and 2.8 $<M_{ee}<$ 3.2 GeV/$c^{2}$, respectively. The $p_{T}$ spectra within STAR acceptance for three selected invariant mass regions (0.4 $<M_{ee}<$ 0.76 GeV/$c^{2}$, 1.2 $<M_{ee}<$ 2.67 GeV/$c^{2}$, and 2.8 $<M_{ee}<$ 3.2 GeV/$c^{2}$) are also reported. They have a fairly sharp transition around 0.1 GeV/$c$ in the most peripheral U + U collisions.

In addition, we report a R\&D work on two different gaseous detectors: mini-drift THick Gas Electron Multiplier chamber (THGEM) and Multi-gap Resistive Plate Chamber (MRPC). This kind mini-drift THGEM chamber is proposed as part of a transition radiation detector (TRD) for identifying electrons and providing additional $dE/dx$ measurement which is essential for small angle scattering at an Electron Ion Collider (EIC) experiment. Through a cosmic ray test, an efficiency larger than 94\% and a spatial resolution $\sim$220 $\mu$m are achieved for the THGEM chamber. Thanks to its outstanding spatial resolution and relative thick ionization gap, the THGEM chamber shows excellent track reconstruction capability. The gain uniformity and stability of the THGEM chamber are also presented. The MRPC is used to upgrade the Beijing Spectrometer end-cap Time of Flight (eTOF) to enhance the particle identification capability. Two different MRPC designs, single-end readout and double-end readout, were tested using 600 MeV/$c$ proton beam at the E3 line of Beijing Electron Positron Collider (BEPC\uppercase\expandafter{\romannumeral2}). The efficiencies of these two different kinds of MRPCs are better than 98\%. The time resolution of double-end readout MRPC is 40 ps while that of the single-end readout MRPC is 47 ps. Moreover, the time resolution of double-end readout MRPC has no incident position dependence while that of single-end readout MRPC has. Since the tracking performance is limited in precision within the Beijing Spectrometer eTOF acceptance, the double-end readout MRPC is selected for the eTOF upgrade according to this beam test. In November 2015, the MRPC-based eTOF system was fully installed in Beijing Spectrometer. The time resolution of the MRPC-based eTOF system obtained from the collision data is $\sim$60 ps, which is much better than the design specification (80 ps).

%\enkeywords{University of Science and Technology of China (USTC), Thesis, Universal \LaTeX{} Template, Bachelor, Master, PhD}
\end{enabstract}
